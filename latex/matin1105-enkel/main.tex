\documentclass[12 pt, a4paper]{article}
\usepackage[norsk]{babel}  								% For norsk oppsett
\usepackage[utf8]{inputenc}
\usepackage{amsmath}
\usepackage{amssymb}
\usepackage{graphicx}
\usepackage{listings}
\usepackage{hyperref}
\usepackage{fancyhdr}
\usepackage{enumerate}
\usepackage{float}
\usepackage{tikz}
\usepackage{circuitikz}
\usepackage{physics}
\usepackage[includeheadfoot, margin =1 in]{geometry}
\usepackage[KJM, OnlyFrontpage]{mnfrontpage} 			%SKIFT HER!!!
\usepackage[version=3]{mhchem}
\usepackage[backend=biber,style=numeric-comp]{biblatex}
\usepackage{siunitx}
\usepackage{todonotes}

\setlength{\parindent}{0cm}

\author{Erik Skaar\\ erikfsk@uio.no}









\begin{document}


\section{}%1 

\subsection*{(a)}


$n \in \mathbb{N}$ then $2^n - (-1)^n$ is divisible by three.


Let's check the first steps: 
\begin{align*}
&n = 1 \qquad 2^1 - (-1)^1 = 3\\
&n = 2 \qquad 2^2 - (-1)^2 = 3\\
&n = 3 \qquad 2^3 - (-1)^3 = 9\\
&n = 4 \qquad 2^4 - (-1)^4 = 15
\end{align*}

All of the first n's are divisible by three. 
We go to general form: 

\begin{align*}
&n \qquad &2^n - (-1)^n\\
&n+1 \qquad &2^{n+1} - (-1)^{n+1} \\
& &= 2(2^{n}) - (-1)^{n+1}\\
& &= 2(2^{n} + (-1)^n  + (-1)^{n+1}) - (-1)^{n+1}\\
& &= 2(2^{n} + (-1)^n  - (-1)^{n}) - (-1)^{n+1}\\
& &= 2(2^{n}- (-1)^{n}) + 2(-1)^n  + (-1)^{n}\\
& &= 2(2^{n}- (-1)^{n}) + 3(-1)^n \\
\end{align*}

Notice that the expression for n+1 is made up by a two part. The first is 2 times the n expression and the second part is 3 times either -1 or +1. Both part are divisible by 3 and there by is the product divisible by 3. 




%\begin{tabular}{|c|c|c|c|c|c|c|}
%	\hline 
%	n & General & Specific & LU & fastest & slowest & $\frac{slowest}{fastest}$\\ 
%	\hline
%	10 & 6.5e-05 & 5e-06 & 4e-05 & Specific & General & 13.0\\ 
%	\hline 
%	100 & 7.5e-05 & 8e-06 & 0.0023 & Specific & LU & 287.5\\ 
%	\hline 
%	1000 & 0.00014 & 4e-05 & 0.26 & Specific & LU & 6500\\ 
%	\hline
%	10000 & 0.0007 & 0.0005 & 142.5 & Specific & LU & 285000 \\ 
%	\hline
%\end{tabular}

%\begin{figure}[H]
%		\centering
%		\includegraphics[width=0.7\linewidth]{ab.png}
%		\caption{Atomene er gule kuler, de elementære vektorene er blå og a vektorene er grønne.}
%		\label{fig:ab}
%\end{figure}


\end{document}
